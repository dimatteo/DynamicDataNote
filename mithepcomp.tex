The MIT High Energy Physics group, part of the MIT Particle Physics Collaboration (PPC), 
currently manages a large computing facility - T2\_US\_MIT - and a smaller computing cluster 
- T3\_US\_MIT - as part of the CMS computing system.

\subsection{The MIT HEP Tier-2}

The MIT HEP Tier-2 (T2\_US\_MIT), located in the MIT-Bates High Performance Research Computing 
Facility (HPRCF), currently hosts about 5000 cores and 3.5 PB of data. The site uplink
speed is 10 Gb/s.

The machines are organized into servers and worker nodes (WNs), which provide a mix of CPU and 
storage space. 

\subsection{The MIT HEP Tier-3}

The MIT HEP Tier-3 (T3\_US\_MIT) located on the MIT campus, currently hosts about 600
cores and 0.3 PB of data. The site uplink speed is 8 Gb/s.

Similar to the Tier-2, the machines are organized into servers and WNs, which provide a mix of CPU and storage 
space. In addition, a significant number of machines, characterized by low-end CPU and memory 
availability, are used as data transfer nodes (TNs).

In the current scheme, T3\_US\_MIT is used to test new services before deployment on the larger site, 
as the Tier-3 hosts only datasets owned by the MIT group, while Tier-2 hosts datasets shared among the 
CMS community, hence downtime and disservices at T2\_US\_MIT will affect a larger community of scientists.



