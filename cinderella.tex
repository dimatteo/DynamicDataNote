The design of Cinderella follows the requirements of an automatic cache release process needed for the 
CMS computing. This cache release process is supposed to optimize the usage of all 
available disk storage and relieve the data managers of a large fraction of their work by:

\begin{enumerate}
	\item keeping all storage filled to a high and safe level;
	\item always allow new data to be received at any site;
	\item remove the least valuable data from the storage when the storage fills over a given level.
\end{enumerate}

In order to perform a cache release, the system is periodically reviewed to ensure there is free space 
for transfers at the site. The condition for the cache release is given by:

\begin{itemize}
	\item usedSpace / availableSpace $>$ 92\% (number can be adjusted);
\end{itemize}

When the cache release algorithm is triggered, enough datasets to restore the target free space are identified
for deletion in order to fulfill the following condition:

\begin{itemize}
	\item usedSpace / availableSpace $<$ 87\% (number can be adjusted);
\end{itemize}

To select datasets for deletion we algorithmically rank all datasets. We select datasets with the lowest priority
by adding the space they occupy until we have enough space to meet the above condition. The selected datasets are 
then removed from storage. The ranking algorithm evaluates the usefulness of the data. For our site, the ranking 
algorithm uses only the time of latest access, such that the lowest priority datasets were accessed longest ago. 

Cinderella implements the algorithm described above. The core parts of the software are
a \textsc{MySQL} database, which stores the list of datasets and relevant information needed to generate the ranking
and perform the cache release, and a daemon that periodically checks if the release trigger condition is met.
In case of a triggered release, the Cinderella daemon issues the deletion command and updates the dataset database.

The Cinderella service also takes periodic snapshots of the dataset database, and uploads them online at 
\href{http://t3serv001.mit.edu/~cmsprod/dynamicdata/cinderella.html}{\textcolor{Mahogany}{this webpage}}.
The snapshots show a ranked list of datasets in our database according to the algorithm used for the
cache release. For each dataset, the number of accesses, number of downloads, caching status (which 
determines if that dataset is present on the Tier-3 cache), last access time and size are listed.
In the current color scheme, blue denotes datasets that were at some point cached but have been deleted and
are not currently stored in cache, red shows the list of datasets that will be deleted in case the cache release
condition is met, and black shows the remaining datasets.

During 6 months of operations, Cinderella has been responsible for deletions that amount to a total of 0.1 PB
of disk space, equivalent to 1/3 of the entire Tier-3 storage capabilities. Equivalently, the system has 
triggered a cache release about ten times.

